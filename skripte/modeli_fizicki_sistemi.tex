\documentclass[10pt]{report}
\usepackage[T1]{fontenc}
\usepackage{geometry}
\usepackage{amsmath}
\usepackage{parskip}

\pagenumbering{gobble}

\newcommand{\question}[1]{\section*{#1}}
\newcommand{\subquestion}[1]{{\large \it #1}}

\linespread{1.2}

% Margine stranice
\newgeometry{
	top=0.25in,
	bottom=0.50in,
	outer=1.0in,
	inner=1.0in
}

% Naslovna strana
\title{\bf Modeli fizičkih sistema \\ (Modeliranje i simulacija sistema)}
\author{Nikola Kušlaković\\\\
https://github.com/nkusla/miss-dump}
\date{Januar 2022.}

\begin{document}
	\maketitle

	\chapter*{Modeli fizičkih sistema}

	\section*{Translatorni mehanički sistemi}

		\textbf{Promenljive} (sve zavise od vremena):
		\begin{itemize}
			\item $x$ - rastojanje $[m]$
			\item $v$ - brzina $[\frac{m}{s}]$, važi $v=\frac{dx}{dt}=\dot{x}$
			\item $a$ - ubrzanje $[\frac{m}{s^2}]$, važi $a=\frac{dv}{dt}=\frac{d^2x}{dt^2}=\ddot{x}$
			\item $f$ - sila $[N]$
		\end{itemize}

		\textbf{Elementi}:
		\begin{itemize}
			\item Masa - mera inertnosti tela $[m]$
			\item Trenje - javlja se kada se dva tela dodiruju i kreću različitim brzinama (npr. trenje podloge, prigušnica...).
			Najčešće opisano linearnom formulom: $f_t = c \cdot \Delta v$, gde je $c$ koeficijent trenja $[\frac{Ns}{m}]$,
			a $\Delta v$ razlika brzina dva tela.
			\item Elastičnost - javlja se kod istegnute opruge. Najčešće opisana linearnom formulom: $f_e = k \cdot \Delta x$, gde je $k$
			koeficijent elastičnosti $[\frac{N}{m}]$, a $\Delta x$ razlika pozicija dva tela.
		\end{itemize}

		\textbf{Zakonitosti}:
		\begin{itemize}
			\item I Njutnov zakon - svako telo teži da ostane u stanju mirovanja ili kretanja, sve dog ga drugo telo ne primora da to stanje promeni.
			\item II Njutnov zakon - sila koja deluje na telo jednaka je proizvodu mase i ubrzanja tog tela.
			$$f = m \cdot a = m \cdot \frac{dv}{dt}$$
			\item III Njutnov zakon - svaka sila akcije na neko telo ima svoju silu reakcije koja je istog itenziteta, ali suprotnog smera.
			\item Dalamberov zakon - suma svih spoljašnjih sila koje deluju na telo i unutrašnje inercijalne sile jednaka je 0.
		\end{itemize}

	\section*{Rotacioni mehanički sistemi}

		\textbf{Promenljive} (sve zavise od vremena):
		\begin{itemize}
			\item $\theta$ - ugao $[rad]$
			\item $\omega$ - ugaona brzina $[\frac{rad}{s}]$, važi $\omega=\frac{d\theta}{dt}=\dot{\theta}$
			\item $\alpha$ - ugaono ubrzanje $[\frac{rad}{s^2}]$, važi $\alpha=\frac{d\omega}{dt}=\frac{d^2\theta}{dt^2}=\ddot{\theta}$
			\item $\tau$ - moment sile $[Nm]$
		\end{itemize}

		\textbf{Elementi}:
		\begin{itemize}
			\item Moment inercije - velicina koja oposuje koliko se telo opire rotacionom kretanju. Zavisi od oblika tela,
			raspodele mase tele i ose oko koje telo rotira. Označava se sa $J$ $[kg m^2]$

			Moment inercije materijalne tačke mase $\Delta m$ koja rotira na rastojanju $r$ od ose je:
			$$J = \Delta m \cdot r^2$$

			\item Trenje usled rotacije - algebarska veza momenta sile i relativne ugaone brzine
			između dve površi koje se dodiruju.
				$$\tau_t = c \cdot \Delta\omega$$
			gde je $c$ - koeficijent trenja $[Nms]$.

			\item Elastičnost usled uvrtanja - algebarska veza momenta sile i relativnog ugaonog pomeraja.
				$$\tau_e = k \cdot \Delta\theta$$
			gde je $k$ - koeficijent elastičnosti $[Nm]$.

			\item Poluga - često uzimamo da je poluga idealna i da nema masu, trenje, moment inercije, unutrašnju energiju.
			Za male pomeraje krajevi poluge se kreću translatorno.
				$$\varphi = \frac{x_1}{L_1} = \frac{x_2}{L_2}$$

			gde je $\varphi$ ugao koji poluga zaklapa sa horizontalom, $x$ je pomeraji, a $L$ dužina dela poluge merene od oslonca poluge.

			\item Zupčanici - uzimamo da su idealni pa onda važi da nemaju moment inercije, trenje, unutrašnju energiju i zubci im savršeno naležu.
			Uvodi se veličina zupčastog prenosa $N$ - odnos broja zubaca.
				$$R_1\theta_1 = R_2\theta_2 \implies \frac{\theta_1}{\theta_2} = \frac{R_2}{R_1} = \frac{z_2}{z_1} = N$$
			gde je $R$ poluprečnik zupčanika, $\theta$ ugaoni pomeraj i $z$ broj zubaca zupčanika.

		\end{itemize}

		\textbf{Zakonitosti}:
		\begin{itemize}
			\item II Njutnov zakon rotacije - moment sile koji deluje na telo jednak je proizvodu momenta inercije i ugaonog ubrzanja tela.
			$$\tau = J \cdot \dot{\omega}$$
			\item Zakon reakcije momenata sila (posledica III Njutnov zakona) - posmatramo dva tela koja rotiraju oko iste ose. Ako momentom sile jadno telo deluje na drugo onda
			i drugo telo momentom sile reakcije deluje na prvo telo istom intenzitetom ali suprotnom smerom.
			\item Dalamberov zakon - isto kao i kod translatornih sistema...
			\item Zakon ugaonih pomeraja - suma razlika ugaonih pomeraja duž zatvorene putanje jednak je 0.
			$$\sum_{i}^{} \theta_i=0 $$
		\end{itemize}

	\section*{Termički sistemi}

		\textbf{Promenljive}:
		\begin{itemize}
			\item $\theta$ - temperatura $[K]$. Najčešće se smatra da je temepratura u svim deolvima tela ista i da je jednaka prosečnoj temperaturi.
			\item $q$ - količina toplote u sekundi $[\frac{J}{s}]=[W]$
		\end{itemize}

		\textbf{Elementi} - 2 pasivna i 1 aktivan element:
		\begin{itemize}
			\item Termička kapacitivnost - daje vezu između temperature tela i akumulirane toplote. Zavisnost se može posmatrati kao linearna:
			$$\dot{\theta}(t) = \frac{1}{C}(q_{in}(t) - q_{out}(t))$$

			$C$ - toplotni kapacitet tela $[\frac{J}{K}]$, računa se po formuli $C = m \cdot \sigma$, gde je $\sigma$ specifična toplota tela.

			\item Termička otpornost - posmatramo provođenje toplote. Provođenje toplote sa jednog tela na drugo telo je srazmerno razlici
			temperatura dva tela. Matematički:
			$$q(t) = \frac{1}{R}(\theta_2(t) - \theta_1(t))$$

			$R$ - termička otpornost $[\frac{Ks}{J}]$, zavisi od karakteristike materijala i računa se po formuli $R = \frac{d}{A\alpha}$,
			gde su $d$, $A$ i  $\alpha$ redom debljina, površina poprečnog preseka i termička provodljivost (podatak iz tabele).

			\item Termički izvor - može biti izvor koji dovodi ili odvodi toplotu iz sistema. Uzimamo da je količina toplote koja
			se dovde pozitivna, a ona koja se odvede negativna.

		\end{itemize}

		\textbf{Zakonitosti}: važe zakoni termodinamike...

	\section*{Sistemi sa fluidima}
		\textbf{Promenljive}:
		\begin{itemize}
			\item q - zapremnski protok $[\frac{m^3}{s}]$
			\item V - zapremina $[m^3]$
			\item h - visina (nivo) tečnosti $[m]$
			\item p - pritisak $[\frac{N}{m^2}] = [Pa]$. Ponekad se posmatra u odnosu na atmosferski pritisak $P_a$
		\end{itemize}

		\textbf{Elementi}:
		\begin{itemize}
			\item Pumpa
			\item Ventil
		\end{itemize}

		\textbf{Zakonitosti}:
		\begin{itemize}
			\item Bernulijeva jednačina
		\end{itemize}

	\section*{Električni i elektromehanički sistemi}
	\textbf{Promenljive}:
	\begin{itemize}
		\item $u$ - napon $[V]$
		\item $i$ - jačina električne struje $[A]$
		\item $f_e$ - elektromagnetna sila $[N]$
		\item $v$ - brzina provodnika u odnosu na magnetno polje $[\frac{m}{s}]$
		\item $l$ - dužina provodnika u magnetnom polju $[m]$
		\item $\phi$ - magnetni fluks $[Wb]$
		\item $B$ -magnetna indukcija $[\frac{Wb}{m^2}]=[T]$
		\item $\epsilon$ - indukovana elektromotorna sila $[V]$
	\end{itemize}

	\textbf{Pasivni elementi}:
	\begin{itemize}
		\item Otpornik otpornosti R $[R]$ - $u(t) = R \cdot i(t)$
		\item Kalem induktinvosti L $[H]$ - $u(t) = L \cdot \frac{di_L(t)}{dt}$
		\item Kondenzator kapaciteta C $[F]$ - $i(t) = R \cdot \frac{du_C(t)}{dt}$
	\end{itemize}
	\textbf{Aktivni elementi}:
	\begin{itemize}
		\item Idealan naponski generator
		\item Idealan strujni izvor
	\end{itemize}

	\textbf{Zakonitosti}:
	\begin{itemize}
		\item Omov zakon
		\item I Kirhofov zakon - algebarska suma struja koje ulaze i izlaze iz čvora jednaka je 0
		\item II Kirhofov zakon - algebarska suma napona po zatvorenoj konturi jednaka je 0
		\item Amperova sila - na pravolinijski provodnik dužine $l$, kroz koji protiče struja jačine $i$ i koji se kreće u homogenom
		magnetnom polju indukcije $B$ deluje sila intenziteta:
			$$f_e = \vec{i}l \times \vec{B} = ilB \cdot sin(\theta)$$
		gde je $\theta$ ugao koje zaklapaju vektori $\vec{i}$ i $\vec{B}$
		\item Indukovanje elektromotorne sile - kretanje provodnika brzinom $v$ u homogenom magnetnom indukuje elektromotornu silu:
			$$\epsilon = \vec{v}l \times \vec{B} = ilB \cdot sin(\theta)$$

	\end{itemize}

\end{document}